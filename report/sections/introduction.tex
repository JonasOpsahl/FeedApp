\section{Introduction}
\label{sec:introduction}

\subsection{Project Overview}
FeedApp makes it easy to gather quick feedback and interaction between users. 
They can create polls, vote, and discuss through comments.
The app solves the problem of gathering input in real time, which often means manual counting or refreshing page for each new vote or comment.
With live server updates, everyone instantly sees new votes and comments as they appear, creating a smooth and interactive experience.
\bigskip

The system is divided into a backend and a frontend.
The backend provides REST endpoints for creating, reading, updating, and deleting polls, options, votes, and comments, and it also sends out events.
The frontend shows the current state, listens to WebSocket updates, and keeps the user interface in sync with the server.
This app is built to test key user interactions, explore event-driven design, and demonstrate a system that can easily scale as more users join.

\subsection{Technology Stack Summary}
The backend is built with Spring Boot, Spring Security, and Java, providing REST APIs, data validation, and integration with messaging and caching systems.
The frontend is made with React and TypeScript, focusing on reusable components and real-time state updates.
An H2 in-memory database is used during development to speed up testing and iteration.
Redis works as a cache and a simple coordination layer for frequently accessed data and precomputed views.
Kafka acts as the messaging backbone, sending and receiving domain events to separate data writes from notifications and later analytics.
Docker and Docker Compose manage all services for a consistent and repeatable setup.
This technology stack was chosen for its scalability (through Kafka and horizontal growth), modularity (clear separation between frontend, backend, cache, and messaging), and maintainability (familiar frameworks and container-based tools).

\subsection{Project Outcome}
The prototype supports creating and managing polls and options, voting with real-time result updates, and a threaded comment system with replies and edits.
WebSocket broadcasting makes sure that votes and comments show up instantly for all users without needing to refresh.
The event-driven setup makes it easy to add new features or services later, while Redis helps speed up repeated data access.
The whole project is containerized for simple local development and easy migration to production databases and clusters.

\subsection{Report Organization}
This report is organized as follows:
Section~\ref{sec:design} presents the overall design and architecture of FeedApp, including key use cases, the domain model, and system components.
Section~\ref{sec:technology} focuses on the featured technology experiment with Kafka, covering its background, hypothesis, and evaluation results, along with supporting technologies such as Redis and Spring Boot with Spring Security.
Section~\ref{sec:implementation} describes the prototype implementation, including backend and frontend details, deployment setup, and integration of Kafka-based event handling.
Finally, Section~\ref{sec:conclusion} summarizes what was learned, discusses the results from Kafka experiment, and possible future improvements.
